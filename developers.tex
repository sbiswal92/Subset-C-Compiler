\documentclass[12pt]{article}

\usepackage{fullpage}
\usepackage{color}
\usepackage{url}
\usepackage{fancyvrb}


\newcommand{\TBD}[1]{{\color{blue}{\bf TBD:} #1}}


\begin{document}

\section{Part 5: Code Generation} 

\url{https://github.com/sbiswal92/Compiler540/releases/tag/Part5}
\\~
\\~This documentation describes the code generation for control flow elements of the language.

\subsection{Features}
\label{Features}
\begin{enumerate}
\item Generates the intermediate code for :
\begin {itemize}
\item if-then
\item if-then-else
\item while
\item do-while
\item for
\item conditional operator \texttt{?:}
\item relational operators \texttt{<,>,<=,>=,==,!=}
\end{itemize}
\item Handles code generation for single-statement body of control-flow elements
\item Handles code generation for multi-statement block of control-flow elements BUT without further local scoping i.e the statement blocks nested inside a function body belong to same scope as body of the function.
\item Short-circuiting (T.B.D)
\end{enumerate}


\subsection{Design}

The parser is a single source file \texttt{parser.y} with dependencies on lexer, symbol table, abstract syntax tree and type checker. The abstract syntax tree is same as Part 4 except new kinds of \texttt{exp\_node} are added to accommodate the control-flow elements.These new kinds include :
\begin{itemize}
\item \texttt{STMT\_COMPOUND}
\item \texttt{STMT\_IF}
\item \texttt{STMT\_IF\_ELSE}
\item \texttt{STMT\_WHILE}
\item \texttt{STMT\_DO\_WHILE}
\item \texttt{STMT\_FOR} etc
\end{itemize}
~
\\The code-generator works in 2 passes as described in Part 4:
\begin{enumerate}
\item Pass 1: To store local variables, global variables, functions and their parameters into the respective symbol table.
\item Pass 2: To generate code for each expression and each control flow element based on the symbols that have been already stored in the symbol tables(s). 
\end{enumerate}


\subsubsection{Code Generation}

The code-generated for conditional statements (in conditionals and loops) is reversed in logic for ease of flow. If an if-condition executes ``true" code upon equality-check pass, then the intermediate code uses inequality-check pass to jump to the ``false" code and otherwise continues to execute the next statement which holds the ``true" code. 

\begin{itemize}
\item \texttt{ <operand1> <relop> <operand2> } :
\begin{verbatim}
	push <operand1>
	push <operand2>
	neg(<relop>)<type> <jump_to_label> 
\end{verbatim}
\end{itemize}

Here, \texttt{neg(<relop>)} is defined as :

\begin{table}[h]
\centering
\begin{tabular}{|c|c|}
\hline
\textless{}relop\textgreater{} & neg(\textless{}relop\textgreater{}) \\ \hline
==                             & !=                                  \\ \hline
!=                             & ==                                  \\ \hline
\textgreater{}=                & \textless{}                         \\ \hline
\textless{}=                   & \textgreater{}                      \\ \hline
\textgreater{}                 & \textless{}=                        \\ \hline
\textless{}                    & \textgreater{}=                     \\ \hline
\end{tabular}
\end{table}


Note : \texttt{<cond\_code>} in the following section logically means \texttt{ifFalse <cond>}
\begin{enumerate}
\item \texttt{ if <cond> then <stmt> } :
\begin{verbatim}
label_false  = new label();

<cond_code> goto label_false;
<stmt_code>;
label_false: ...
\end{verbatim}


\item \texttt{ if <cond> then <stmt1> else <stmt2> } :
\begin{verbatim}
label_false  = new label();
label_exit  = new label();

<cond_code> goto label_false;
<stmt1_code>;
goto label_exit;
label_false:
<stmt2_code>;
label_exit: ...
\end{verbatim}


\item \texttt{ while <cond> <stmt> } :
\begin{verbatim}
label_false  = new label();
label_cond = new label();

label_cond:
<cond_code> goto label_false;
<stmt_code>;
goto label_cond;
label_false: ...
\end{verbatim}


\item \texttt{ do <stmt> while <cond> } :
\begin{verbatim}
label_false  = new label();
label_cond = new label();
label_loop = new label();

goto label_loop;
label_cond:
<cond_code> goto label_false;
label_loop:
<stmt_code>;
goto label_cond;
label_false: ...

\end{verbatim}


\item \texttt{ for <assign;cond;update>  <stmt> }
\begin{verbatim}
label_false  = new label();
label_cond = new label();

<assign_code>;
label_cond:
<cond_code> goto label_false;
<stmt_code>;
<update_code>;
goto label_cond;
label_false: ...

\end{verbatim}

\item \texttt{ <cond> ? <stmt1> : <stmt2> } : This operation can r.h.s of an assignment statement or a standalone expression depending upon type of \texttt{exp\_node}, \texttt{<stmt1>} and \texttt{<stmt2>} are.
If \texttt{<stmt1>} and \texttt{<stmt2>} are identifier or literal :  
\begin{verbatim}
label_false  = new label();
label_exit  = new label();

<cond_code> goto label_false;
push(v) <stmt1_code>;
goto label_exit;
label_false:
push(v) <stmt2_code>;
label_exit: ...
\end{verbatim}

Otherwise :  
\begin{verbatim}
label_false  = new label();
label_exit  = new label();

<cond_code> goto label_false;
<stmt1_code>;
goto label_exit;
label_false:
<stmt2_code>;
label_exit: ...
\end{verbatim}
\end{enumerate}

\subsubsection{Short Circuit Generation}
This section describes the short-circuit logics injected into above code-generation to prevent unnecessary condition checks.

\begin{enumerate}
\item
\end{enumerate}

\end{document}